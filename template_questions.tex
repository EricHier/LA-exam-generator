\documentclass[addpoints]{exam}
\usepackage[utf8]{inputenc}
\usepackage[german]{babel}
\usepackage{amsmath} 
\usepackage{amssymb}

\pointpoints{Punkt}{Punkte}

\pagestyle{headandfoot}
\runningheadrule
\firstpageheader{Lineare Algebra}{Probe Klausur}{\today}
\runningheader{Lineare Algebra}{Probe Klausur, Seite \thepage\ von \numpages}{\today}

\renewcommand{\solutiontitle}{\noindent\textbf{Solution:}\par\noindent}

{{PREAMBLE}}

\begin{document}

\vspace{0.2in}
Name:\enspace\hrulefill
\hspace{0.2in}
Matrikel Nummer:\enspace\hrulefill

\vspace{0.2in}

\begin{center}
    \centering
    Das Blat zeigt ihnen typische Aufgaben, die in der Klausur vorkommen könnten. Der Umfang an
    Rechenaufgaben und schriftliche zu bearbeitenden Aufgaben entspricht ungefähr der Klausur.
    In der Klausur haben Sie {{TIME}} Minuten Zeit für die Bearbeitung.
    Zum Bestehen der Klausur müssen Sie mindestens die Hälfte der maximal erreichbaren Punkte erreichen.
\end{center}

\vspace{0.2in}

{{QUESTIONS}}

\end{document}

